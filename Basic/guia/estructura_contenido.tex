\documentclass{article}
\usepackage[utf8]{inputenc}
\usepackage[spanish]{babel}
\usepackage{soul}
\usepackage{ulem}
\usepackage{graphicx}
\usepackage{float}
\usepackage{subfigure}
\usepackage{multicol}
\usepackage{pifont}
\usepackage{enumerate}
\usepackage{amsmath}
\title{Estructura de contenido}
\author{Daniel Ortiz}
\date{December 2018}
\begin{document}
\maketitle
\newpage
\tableofcontents
\newpage
\section{Paquetes de idiomas}
Ejemplo de tilde (cuando no se detecta de forma normal):
\begin{center}
    El rat\'on conduce un camión
\end{center}
Ejemplo de texto entrecomillado (cuando no se detecta de forma normal):
\begin{center}
     ``Esto es una cita''
\end{center}
Signos de interrogaciones (cuando no se detecta de forma normal):
\begin{center}
     ?`Esto es una pregunta?
\end{center}
\clearpage
\section{Inserción de imágenes}
\subsection{Inserción básica}
\begin{figure}[H] %con la posicion H indicamos que colocamos la imagen justo donde aparece en el código
\centering \includegraphics[width=0.5\textwidth]{Images/image1.png} \caption{Imagen de ejemplo}
\end{figure}
\begin{figure}[h]
\raggedright \includegraphics[width=3cm]{Images/image2.png} 
\caption{Imagen con distintos parámetros}
\end{figure}
\subsection{Inserción conjunta}
\begin{figure}[H] 
\centering
\subfigure[Image 1]{\includegraphics[width=40mm]{Images/image1.png}}
\subfigure[Image 2]{\includegraphics[width=40mm]{Images/image2.png}}
\caption{Dos imágenes juntas con el mismo ancho} 
\end{figure}
\begin{multicols}{2} 
    \begin{figure}[H]
    \centering \includegraphics[width=0.4\textwidth]{Images/image1.png} \caption{Image 1 enfrentada}
    \end{figure} 
    \begin{figure}[H]
    \centering \includegraphics[width=0.4\textwidth]{Images/image2.png} \caption{Image 2 enfrentada}
    \end{figure} 
\end{multicols}
\begin{multicols}{2}
\begin{figure}[H]
\centering
\includegraphics[width=0.3\textwidth]{Images/image1.png}
\caption{Imagen con texto}
\end{figure}
Lorem ipsum dolor sit amet, consectetur adipiscing elit, sed do eiusmod tempor incididunt ut labore et dolore magna aliqua. Ut enim ad minim veniam, quis nostrud exercitation ullamco laboris nisi ut aliquip ex ea commodo consequat.
\end{multicols}
\clearpage
\section{Tablas}
\subsection{Tabla básica con tabbing}
\begin{tabbing}
%hspace indica la anchura de las columnas y\kill que esta fila no se imprime
\hspace*{3 cm} \= \hspace*{6 cm}\= \hspace*{3cm}\kill
% > indica que se pasa al siguiente tab de la tabla y \\ que pasamos de fila
Nombre \> DNI \> Nota final\\
Luis \> 52047692Q \> 8.56\\
Nuria \> 0285801L \> 9.12\\
Esteban \> 145058R \> 5.2\\
Lucia \> 7852205E \> 2.3
\end{tabbing}
\subsection{Tabla avanzadas con tabular}
% tabla creada con https://www.tablesgenerator.com/
\begin{table}[H]
\begin{tabular}{|c|c|c|c|c}
\cline{1-4}
\textbf{Columna 1} & \textbf{Columna 2} & \textbf{Columna 3} & \textbf{Columna final} &  \\ \cline{1-4}
Valor 1.1          & Valor 1.2          & Valor 1.3          & Valor Final 1          &  \\ \cline{1-4}
Valor 2.1          & Valor 2.2          & Valor 2.3          & Valor Final 2          &  \\ \cline{1-4}
Valor 3.1          & Valor 3.2          & Valor 3.3          & Valor Final 3          &  \\ \cline{1-4}
\end{tabular}
\end{table} 
\newpage
\section{Items} 
\subsection{Items básicos}
\begin{itemize}
    \item Items de primer nivel
        \begin{itemize}
            \item Items de segundo nivel 
            \item Items de segundo nivel
            \begin{enumerate}
            \item Items enumerados de tercer nivel
            \item Items enumerados de tercer nivel
            \end{enumerate}
            \item Items de segundo nivel
               \begin{itemize}
                    \item Items de tercer nivel 
                    \begin{itemize}
                        \item Existen hasta items de cuarto nivel
                    \end{itemize}
               \end{itemize}
        \end{itemize}
    \item Items de primer nivel
\end{itemize}
\subsection{Modificando marcadores}
% los distintos marcadores se obtienen de la url http://willbenton.com/wb-images/pifont.pdf
% el numero resultante sale de sumar el numero de la fila que queremos mas la columna
\begin{itemize} \renewcommand{\labelitemi}{\ding{51}}
    \item Caso 1. 
    \item Caso 2. 
    \item Caso 3.
    \begin{itemize} \renewcommand{\labelitemii}{\ding{60}}
        \item \textbf{Figura 1}.
        \item \textbf{Figura 2}.
            \begin{itemize} \renewcommand{\labelitemiii}{\ding{81}}
            \item Ejemplo 1. 
            \item Ejemplo 2. 
            \item Ejemplo 3. 
            \begin{itemize}
                \renewcommand{\labelitemiv}{\ding{248}} 
                \item Afirmativo.
                \item Negativo. 
            \end{itemize}
        \end{itemize} 
    \end{itemize}
\end{itemize}
\subsection{Enumerados de más nivel}
\begin{enumerate} 
    \item Ejemplo 1. 
    \item Ejemplo 2.
    \begin{enumerate} 
        \item Caso 1. 
        \item Caso 2. 
        \item Caso 3.
        \begin{enumerate} 
            \item Figura 1. 
            \item Figura 2.
            \begin{enumerate}[I] 
                \item Ecuacion 1. 
                \item Ecuacion 2. 
            \end{enumerate} 
        \end{enumerate}
    \end{enumerate} 
\end{enumerate}
\section{Modo matemático}
\subsection{Fórmulas simples}
Ejemplo de fórmula sencilla:
\[
y=mx+b
\]
\newpage
\section{Bibliografía, enlaces y referencias} \label{sec:Seccion 1}
\subsection{Referencias dentro del documento} 
\begin{flushleft}
La sección \ref{sec:Seccion 1} trata referencias.
\end{flushleft}
\subsection{Bibliografía}
\subsubsection{Bibliografía simple}
\begin{thebibliography}{}
\bibitem{ej1}\textsc{Autor 1}
\textit{Tema 1} Título de artículo 1
\bibitem{ej2}\textsc{Apellido Autor 2, Nombre}
\textit{Tema 2}, 2019
\end{thebibliography}
\subsubsection{Google Scholar y BibTex}
\noindent 
De esta manera podemos nombrar nuestra bibliografia\cite{wood2014ethereum}. \\
De esta manera podemos añadir otra cita\cite{mian2013provisioning}.
\bibliographystyle{acm}
\bibliography{bibliografia}
\end{document}