\documentclass{article}
\usepackage[utf8]{inputenc}
\usepackage{soul}
\usepackage{ulem}
\title{Ejemplos de sintaxis}
\author{Daniel Ortiz}
\date{December 2018}
\begin{document}
\maketitle
 \newpage
\section{Saltos de página}
Saltamos de página \newpage
Volvemos a saltar de página pero acabando todos los elementos flotantes que hubiera \clearpage
\section{Saltos de línea}
Salto de línea básico 1. \\
Salto de línea con espaciado.\par
Salto de línea básico 2  \quad  y espaciado. \newline
\section{Alineaciones de texto}
\begin{flushleft}
Texto a la izquierda.
\end{flushleft}
\begin{center}
Texto centrado.
\end{center}
\begin{flushright}
Texto a la derecha.
\end{flushright}
\section{Modificadores de texto}
\subsection{Básicos}
\textbf{Texto en negrita}, y normal.
\newline
\newline
\textit{Texto en cursiva}, y normal.
\newline
\newline
\texttt{Texto en maquina de escribir}, y normal.
\newline
\newline
\textsc{Texto en versálita}, y normal.
\newline
\newline
\emph{Texto emfatizado}, y normal.
\newline
\newline
\underline{Texto subrayado}, y normal.
\newline
\newline
\textrm{Texto redondo}, y normal.
\begin{verbatim}
Texto por consola y terminal
\end{verbatim}
y normal.
\newline
\newline
\textst{texto tachado 1}, y normal. %necesita el paquete soul
\newline
\newline
\sout{texto tachado 2}, y normal. %necesita el paquete ulem
\newline
\newline
\uuline{texto doble tachado}, y normal. %necesita el paquete ulem
\newline
\newline
\uwave{texto tachado ondulado}, y normal. %necesita el paquete ulem
\newline
\newline
\xout{texto tachado con impetú}, y normal. %necesita el paquete ulem
\clearpage
\subsection{Tamaño de letra}
\normalsize{Letra con tamaño normalizado}, y normal.
\newline
\newline
{\tiny{Letra con tamaño diminuto}}, y normal.
\newline
\newline
{\scriptsize{Letra muy pequeña}}, y normal.
\newline
\newline
{\footnotesize{Letra pequeña}}, y normal.
\newline
\newline
{\small{Letra más grande de las pequeñas}}, y normal.
\newline
\newline
{\large{Letra grande}}, y normal.
\newline
\newline
{\Large{Letra más grande}}, y normal.
\newline
\newline
{\LARGE{Letra muy grande}}, y normal.
\newline
\newline
{\huge{Letra enorme}}, y normal.
\newline
\newline
{\Huge{Letra más enorme}}, y normal.
\subsection{Ejemplos de combinaciones}
\begin{center}
Frase con \small{\textit{itálica pequeña}} en una estructura centrada.
\end{center}
Frase con {\Huge{\textsc{versálita enorme}}}.
\begin{flushleft}
Frase con letra {\tiny{\texttt{pequeña y a máquina}}} en una estructura a la izquierda.
\end{flushleft}
\end{document}